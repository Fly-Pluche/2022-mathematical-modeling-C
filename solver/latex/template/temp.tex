\documentclass[UTF8]{ctexart}

\usepackage{listings}
\usepackage{color,xcolor} 
\usepackage{colortbl}
\usepackage{graphicx}
\usepackage{booktabs} %绘制表格
\usepackage{caption2} %标题居中
\usepackage{geometry}
\usepackage{array}
\usepackage{amsmath}
\usepackage{subfigure} 
\usepackage{longtable}
\usepackage{abstract}
\usepackage{multirow}
\usepackage{enumerate}
\usepackage{float}
\usepackage{type1cm}
%for long table
\usepackage{longtable}

%for table toprule line
\usepackage{booktabs}
% \usepackage{zhnumber} % change section number to chinese
% \renewcommand\thesection{\zhnum{section}}
% \renewcommand \thesubsection {\arabic{section}}


%伪代码
\usepackage{algorithm}
\usepackage{algpseudocode}
\usepackage{amsmath}
\renewcommand{\algorithmicrequire}{\textbf{输入:}}  % Use Input in the format of Algorithm
\renewcommand{\algorithmicensure}{\textbf{过程:}} % Use Output in the format of Algorithm

\usepackage{xltxtra}
\usepackage{mflogo,texnames}
\usepackage{amssymb}

%附录代码
\usepackage{listings}
\usepackage{xcolor}

\lstset{
    numbers=left, 
    numberstyle= \tiny, 
    keywordstyle= \color{ blue!70},
    commentstyle= \color{red!50!green!50!blue!50}, 
    frame=shadowbox, % 阴影效果
    rulesepcolor= \color{ red!20!green!20!blue!20} ,
    escapeinside=``, % 英文分号中可写入中文
    xleftmargin=2em,xrightmargin=2em, aboveskip=1em,
    framexleftmargin=2em
} 

\pagestyle{plain} %页眉消失

\geometry{a4paper,left=2.5cm,right=2.5cm,top=2.5cm,bottom=2.5cm}%设置页面尺寸
\lstset{
		numbers=left, %设置行号位置
		numberstyle=\tiny, %设置行号大小	
		keywordstyle=\color{blue}, %设置关键字颜色
		commentstyle=\color[cmyk]{1,0,1,0}, %设置注释颜色
		escapeinside=``, %逃逸字符(1左面的键),用于显示中文
		breaklines, %自动折行
		extendedchars=false, %解决代码跨页时,章节标题,页眉等汉字不显示的问题
		xleftmargin=1em,xrightmargin=1em, aboveskip=1em, %设置边距
		tabsize=4, %设置tab空格数
		showspaces=false %不显示空格
	}


		% \begin{figure}[!htbp]\centering
		% \includegraphics[width=1\textwidth]{img} % 图片相对位置
		% \label{fig:figure 0} % 图片标签
		% \end{figure}


\title{\textbf{标题}}
% \date{August 24, 2022}
\date{} %关闭日期时间的显示

\begin{document}
\maketitle{}
\renewcommand{\abstractname}{\Large 摘要\\}   %摘要部分
\begin{abstract}
	\normalsize
    \CJKfamily{song}
    \fontsize{12pt}{18pt}
	
    % 摘要内容
    % 摘要内容
    % 摘要内容
    % 摘要内容


	\textbf{关键字}: ;  ;  ;  ;  ;   %关键词
\end{abstract}
\thispagestyle{empty}    %该页不编码且独立为一页
\newpage

\setcounter{page}{1}  %此页开始编码

%第一部分
\section{问题重述}

\subsection{问题背景}


\subsection{问题提出}






%第二部分
\section{问题分析}
\subsection{问题一的分析}




\subsection{问题二的分析}




\subsection{问题三的分析}




\subsection{问题四的分析}


%第三部分
\section{模型假设}
1.假设......................

2.

3.

4.

5.

6.

%第四部分
\section{符号说明}


		\begin{table}[H] 
		\begin{center}  
		\begin{tabular}{c|c|c}    
		\toprule[2pt]    
		\rowcolor[gray]{0.8}

		\multicolumn{1}{m{8em}}{\centering 符号}  &\multicolumn{1}{m{15em}}{\centering 基本说明} &\multicolumn{1}{m{10em}}{\centering 单位}\\

		%直接用合并单元格的方法来实现自定义列宽的同时,使文字居中对齐

		\midrule[1.3pt]
		    $\alpha$                                & x        & | \\
			$\lambda$                                & x        & |       \\
			$\theta$                                 & x        & |            \\


		\bottomrule[2pt]   
		\end{tabular}  
		\end{center}
		\end{table}




%第五部分
\section{问题一的模型建立与求解}
\subsection{问题分析}



\subsubsection{模型建立}



\subsubsection{模型求解}

\textbf{$step1:$}

\textbf{$step2:$}

\textbf{$step3:$}

\textbf{$step4:$}






%第六部分
\section{问题二的模型建立与求解}
\subsection{问题分析}



\subsubsection{模型建立}



\subsubsection{模型求解}

\textbf{$step1:$}

\textbf{$step2:$}

\textbf{$step3:$}

\textbf{$step4:$}







%第七部分
\section{问题三的模型建立与求解}
\subsection{问题分析}



\subsubsection{模型建立}



\subsubsection{模型求解}

\textbf{$step1:$}

\textbf{$step2:$}

\textbf{$step3:$}

\textbf{$step4:$}


%第八部分
\section{问题四的模型建立与求解}
\subsection{问题分析}



\subsubsection{模型建立}



\subsubsection{模型求解}

\textbf{$step1:$}

\textbf{$step2:$}

\textbf{$step3:$}

\textbf{$step4:$}

%第九部分
%此处看情况,多少个模型写多少个评价
\section{模型的评价}
\subsection{模型一}
\textbf{优点:}


\textbf{缺点:}

\subsection{模型二}
\textbf{优点:}


\textbf{缺点:}

\subsection{模型三}
\textbf{优点:}


\textbf{缺点:}

\subsection{模型四}
\textbf{优点:}


\textbf{缺点:}


%第十部分
\section{参考文献}
\noindent{[1]姜启源,谢金星,叶俊,数学模型(第四版),北京:高等教育出版社,2011.1}

\noindent{[2] 司宛玲,司守奎,数学建模简明教程[M]北京:国防工业出版社,2019,185-189}

\noindent{[3] 韩中庚,数学建模方法及其应用(第二版)[M]北京:高等教育出版社,2010,163-171}
\noindent{}
\noindent{}













\newpage %参考文献完后重启一页




% \clearpage
% \bibliographystyle{plain}
% \bibliography{ref}%ref指向自己创建的ref.bib
% % IoU\cite{zheng2020distance}
% \clearpage

%最后一部分
\section{附录}
\subsection{代码}

\lstset{language=python}
\begin{lstlisting}
	
\end{lstlisting}


\end{document}
